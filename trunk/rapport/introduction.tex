\section{Introduction}
L'intérêt de ce bureau d'étude était double : celui-ci devait nous permettre de mettre en application
tout ce que nous avons appris du langage C mais aussi d'apprendre à travailler de A à Z sur l'élaboration
d'un programme bien défini. Cela permet entre autres de se familiariser avec certains outils de la programmation, quitte
à se lancer dans des rechers personnelles, mais cela apprend aussi à savoir diviser les tâches lors de l'écriture
d'un programme, en fonction des compétences de chacun.
La réalisation du "Zuma", la diversité des points à respecter ainsi que la marge de manoeuvre qui nouq était laissée ont donc été fidèles aux objectifs d'un travail de programmation.   

